Human metabolism plays a key role in disease pathogenesis and drug action. Half a century of biochemical literature leveraged by the advent of genomics allowed the emergence of computational modeling techniques and the \textit{in silico} analysis of complex biological systems. In particular, Constraint-Based Reconstruction and Analysis (COBRA) methods address the complexity of metabolism through building tissue-specific networks in their steady state. It is known that biological systems respond to perturbations induced by pathogens, drugs or malignant processes by shifting their activity to safeguard key metabolic functions. Extending the modeling framework to consider the dynamics of these  complex systems will bring simulations closer to observable human phenotypes. 
In this thesis, I combined physiologically-based pharmacokinetic (PBPK) models with genome-scale metabolic models (GSMMs) to form hybrid genome-scale dynamical models that provide a hypothesis-free framework to study the perturbations induced by one or more perturbagen on human tissues. On a first stage, these methodologies were applied to decipher the absorption of levodopa and amino acids by the intestinal epithelium and allowed to derive a model-based diet for Parkinson's Disease patients. In the next phase, we extended the study to 605 drugs in order to predict the occurrence of gastrointestinal side effects through a machine learning classifier, using a combination of gene expression and metabolic reactions set as features. Finally, the approach upscaled to several tissues, specifically to investigate the genesis of metabolic symptoms in type 1 diabetes and to suggest key metabolic players underlying within and between-individual variability to insulin action. Taken as whole, the integration of two modeling techniques constrained by expert biological knowledge and heterogeneous data types will be a step forward in achieving convergence in human biology.